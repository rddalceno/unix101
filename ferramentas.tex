
%%% Local Variables: 
%%% mode: latex
%%% TeX-master: unix-101
%%% End: 

\chapter{Ferramentas}

Ao longo do tempo, diversas ferramentas foram desenvolvidas para sistemas
Unix. Algumas dessas ferramentas se transformaram em padrão de
mercado. Outras, são versões melhoradas ou ``concorrentes'' das primeiras.

\begin{enumerate}
\item{Interpretadores de Comandos}

\end{itemize}


\item{Processadores de Texto}

Hoje, o uso de editores WYSIWYG é comum e escrever documentos de outra forma é
quase impensável (especialmente se voce não for um usuário Unix). Mas nem
sempre foi assim. Se voce era um feliz usuário de MS-Windows
3.1, no início da década de 1990, é bem provável que tenha usado o MS-Word em uma de suas primeiras
versões. Ainda hoje, é uma das principais referências em editores WYSIWYG.

Antes disso, editores de texto eram apenas editores de texto. Se você quisesse
um documento impresso com qualidade, precisava usar também um processador de
texto. O processador de texto funciona de modo parecido a um compilador. Ele
lê o texto e, de acordo com marcas de formatação e comandos especiais
inseridos no texto, o processador, então, usa essas informações para
renderizar o documento final.

\begin{itemize}
\item{Roff e derivados}

Roff é uma ferramenta de processamento e formatação de texto. Em meados da
decada de 1960, os computadores ainda eram pouco usados para essa
finalidade. Mas quando eram, o faziam de forma terrivelmente complicada.


\item{Postscript e Ghostscript}

\item{\TeX\ ,\LaTeX\ e \LaTeXe}


\end{itemize}

\item{Paginadores}
\begin{itemize}

\item{More}

\item{Less}

\end{itemize}
\end{enumerate}
