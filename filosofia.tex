
%%% Local Variables: 
%%% mode: latex
%%% TeX-master: t
%%% End: 
\chapter{Filosofia}
Ao longo do tempo, voces ouvirão sobre filosofia Unix, se já não tiverem
ouvido. Neste capítulo, pretendo apresentar essa filosofia de modo que voces
possam entender melhor como um sistema Unix funciona, como suas aplicações
devem funcionar e, quando tiverem que escrever alguma ferramenta 

\begin{enumerate}
\item{\emph{Free as in Freedom}}

Em seus primórdios, o desenvolvimento e utilização do Unix estava restrito a
Universidades e centros de pesquisa. Exatamente por isso, havia um espírito
colaborativo, onde os usuários criavam novos códigos e compartilhavam
abertamente esses códigos com outros usuários.

Em um certo momento, as empresas perceberam o valor comercial desse novo
sistema e começaram a comercializá-lo. Para isso, resolveram que seu código
não poderia mais ser compartilhado livrementee ,assi, começaram a surgir

\item{KISS}

Mantenha isso simples. Cedo ou tarde você terá que criar suas próprias
ferramentas para facilitar o seu tarbalho e, talvez, o dos seus companheiros.
Por isso, mantenha isso simples. 

Essa é uma das principais premissas de desenvolvimento do Unix, de suas
ferramentas e de sua operação.

\item{RTFM}

Leia o manual. Quando tudo o mais falhar, leia o manual. Se tiver dúvidas
sobre a sintaxe correta, opções de uso ou até mesmo sobre o que faz uma
determinada ferramenta, leia o manual.

Alguém gastou muito tempo fazendo essa documentação. Convenhamos, escrever
documentação é um trabalho árduo e, desgastante e, muito vezes, chato. E mesmo
assim, alguém dedicou muitas e muitas horas nisso. Portanto, antes de
perguntar algo em uma lista de email ou fórum na internet, leia o manual da
ferramenta. Isso vai lhe poupar tempo e (muito provavelmente) alguns insultos
por parte de outros usuários menos sociáveis.

\end{enumerate}

