
%%% Local Variables: 
%%% mode: latex
%%% TeX-master: t
%%% End: 
\chapter{História}
Em 1965, em uma conferência de computação, foi apresentado o projeto de um novo sistema utilitário chamado MULTICS. Como seria executado em Mainframes, o Multics foi projetado como um utilitário de proposito geral de compatilhamento de tempo.

Inicialmente, participaram do projeto General Eletric, M.I.T e Bell Labs, mas
a Bell Labs deixou o consórcio logo no início do projeto. 

O Multics tornou-se um produto comercial da G.E., que vendia serviços de
compartilhamento de tempo (time-sharing) em mainframes.
