
%%% Local Variables: 
%%% mode: latex
%%% TeX-master: "unix-101"
%%% End: 

\chapter{Editores de Textos }

Uma ferramenta essencial para um usuário de qualquer sistema operacional é o
editor de textos.

\begin{enumerate}
\item{Ed}

Ed era um editor de linha de comando, projetado para funcionar nos teletipos
para Unix. O nome \emph{ed} vem da abreviação de
\emph{editor}. Não era muito amigável, quase sem recursos, mas foi a
ferramenta padrão para edição de textos por muito tempo, até que Bill Joy
resolveu escrever um novo editor, com mais recursos e (um pouco) mais
amigável chamado...

\item{Vi}

Bill Joy e Chuck Haley escreveram um novo editor chamado \emph{ex}, pois o
\emph{ed} já não era suficiente para utilização em um monitor.

Seu nome vem de \em{VIsual editor}. 

\item{Vim}

\item{Emacs}

Criado por Richard Stallman, é parte essencial do projeto GNU.
\end{enumerate}
