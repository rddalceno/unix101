
%%% Local Variables: 
%%% mode: latex
%%% TeX-master: "unix-101"
%%% End: 
\chapter{Sabores}

O compartilhamento de informações e códigos era algo comum entre desenvolvedores e
pesquisadores durante o final da decada de 1960 até o final da década de 1970.

Essa cultura de compartilhamento de informações fez com que diversas veroes
diferentes do Unix surgissem. Isso ficou mais evidente quando a AT\&T decidiu
que sua versão não poderia mais ser copiada e redistribuída livremente, fzendo
dele um sistema proprietário.

A AT\&T viu o potencial desse novo sistema operacional e passou a
comercalizá-lo. As universidades não podiam pagar pela licença do unix e,
ainda mais importante, não podiam fazer cópias dele.  Para a Universidade da
Califórnia em Berkeley, isso não era aceitavel. Ao menos, não para os
estudantes e pesquisadores que vivam lá. Esses estudantes e pesquisadores
resolverram, então, desenvolver sua própria versão do Unix, com uma licença
que permitia a cópia, redistribuição e alteração do código fonte. Essa foi a
primeira versão do \emph{Berkeley Software Distribution} ou simplesmente BSD.

Muitas versões diferentes acabariam derivando desse trabalho inicial dando
origem ao 386BSD, PC-BSD, FreeBSD, NetBSD, OpenBSD e assim por diante.

Da AT\&T surgiu o Unix System V. Alguns grandes players da coputação criaram
suas próprias versões do Unix também. Alguns se basearam no System V da At\&T,
outros, no BSD. Alguns derivados do System V são o DEC Tru64, Solaris, QNX e Linux.

A essaa miríade de versões e fornecedores diferentes, damos o nome de
\emph{sabores} ou \emph{flavors}.
